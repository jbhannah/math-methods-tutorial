\chapter{Introduction}

This is not a math textbook. Instead, this tutorial is designed to be used as a
guide for physics students as they begin to navigate the world of post-calculus
mathematics. Core concepts and equations will be introduced, which will be
essential to the study of classical and quantum mechanics, thermodynamics, and
other upper-division physics subjects; in combination with other resources, the
introduction in this tutorial will provide a jumping-off point for students to
develop a sufficient understanding of the mathematics behind the physics.

Such resources include:

\begin{itemize}

\item A math methods textbook, such as George Arfken's excellent
\textit{Mathematical Methods for Physicists}. After reading the chapter on a
topic in this tutorial, read the related sections of the textbook to further
study, practice, and internalize the concepts and techniques introduced in the
tutorial.

\item The NIST Digital Library of Mathematical Functions (DLMF), available at
\url{http://dlmf.nist.gov/}. In particular, this will be a useful reference in
the study of Bessel functions, Legendre and associated functions, and spherical
harmonics.

\item The solutions included in the Appendix~\ref{app:solutions} of this book.
These are provided on the honor system, trusting the student to only refer to
them to verify their own answers to the exercises in each chapter. Hints are
also provided for each exercise, which should be used as the first resort in the
case of not knowing where to begin on that exercise.

\item If this book is being used in a class, the teacher, teaching assistant,
and fellow students. Individual instruction and assistance, as well as peer
study, are as valuable to the student as any amount of reference material.

\end{itemize}

As the student uses the mathematical concepts introduced in this tutorial, it
can also be used as a reference manual to refresh their knowledge of those
concepts. Taking the time to match the physical concepts taught in
upper-division physics courses with their mathematical abstractions can grant
the student a greater understanding of both as they progress towards earning
their degree.
